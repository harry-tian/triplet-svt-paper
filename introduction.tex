%!TEX root = main.tex

\section{Introduction}

In recent years machine learning and AI has seen impressive development and drew great attention, from ML models surpassing human classification benchmarks to neural networks generating realistic text and even human faces. However, (and quite reasonably) ML models has shown less proficiency in learning or aligning with human values. In this work, we investigate human alignment in the form of human similarity judgements: given the question "Is \textit{a} more similar to \textit{b} than to \textit{c}?”, human provide feedback that can be interpreted as a constraint $dist(a, b) < dist(a, c)$. Learning such a perceptual distance metric or an embedding that respects the constraint can be useful in tasks that involve human-AI collaboration. 

The problem of learning human similarity judgment is not a novel one. There is abundant work on solving this problem, such as Stochastic Triplet Embedding (STE) \citep{2012tste}, Crowd Kernel Learning \citep{2011CKL} active multidimensional scaling \citep{2011aMDS}, and many more \citep{roads2020psiz,chen2019loe}. As ~\cite{chen2019loe} noted, all of these methods involve expensive computations and face scalability problems. This work does not aim to solve those issues or propose an algorithm surpassing the state-of-the-art, but instead fall back to the simplest (and perhaps more interpretable) matrix completion methods. From analyzing the performance of simpler methods we hope to find quantitative or qualitative patterns in human similarity judgment.

In this work we implement the singular value thresholding algorithm \citep{2008svt} as our matrix completion method. Given human similarity judgments in the form of triplet annotations, we experiment with different distance matrix constructions methods. We find that overall, matrix produced by SVT perform better than our baselines. We also experiment with varying number of data and find that...
