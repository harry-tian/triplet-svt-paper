%%%%%%%% ICML 2022 EXAMPLE LATEX SUBMISSION FILE %%%%%%%%%%%%%%%%%

\documentclass[nohyperref]{article}

% Recommended, but optional, packages for figures and better typesetting:
\usepackage{microtype}
\usepackage{graphicx}
% \usepackage{subfigure}
\usepackage{booktabs} % for professional tables

% hyperref makes hyperlinks in the resulting PDF.
% If your build breaks (sometimes temporarily if a hyperlink spans a page)
% please comment out the following usepackage line and replace
% \usepackage{icml2022} with \usepackage[nohyperref]{icml2022} above.
\usepackage{hyperref}


% Attempt to make hyperref and algorithmic work together better:
\newcommand{\theHalgorithm}{\arabic{algorithm}}

% Use the following line for the initial blind version submitted for review:
\usepackage{icml2022}

% If accepted, instead use the following line for the camera-ready submission:
% \usepackage[accepted]{icml2022}

% For theorems and such
\usepackage{amsmath}
\usepackage{amssymb}
\usepackage{mathtools}
\usepackage{amsthm}

% if you use cleveref..
\usepackage[capitalize,noabbrev]{cleveref}

%%%%%%%%%%%%%%%%%%%%%%%%%%%%%%%%
% THEOREMS
%%%%%%%%%%%%%%%%%%%%%%%%%%%%%%%%
\theoremstyle{plain}
\newtheorem{theorem}{Theorem}[section]
\newtheorem{proposition}[theorem]{Proposition}
\newtheorem{lemma}[theorem]{Lemma}
\newtheorem{corollary}[theorem]{Corollary}
\theoremstyle{definition}
\newtheorem{definition}[theorem]{Definition}
\newtheorem{assumption}[theorem]{Assumption}
\theoremstyle{remark}
\newtheorem{remark}[theorem]{Remark}

% Todonotes is useful during development; simply uncomment the next line
%    and comment out the line below the next line to turn off comments
%\usepackage[disable,textsize=tiny]{todonotes}
\usepackage[textsize=tiny]{todonotes}


% HAN: Import from NeurIPS templates
\usepackage[utf8]{inputenc} % allow utf-8 input
\usepackage[T1]{fontenc}    % use 8-bit T1 fonts
\usepackage{url}            % simple URL typesetting
\usepackage{booktabs}       % professional-quality tables
\usepackage{amsfonts}       % blackboard math symbols
\usepackage{nicefrac}       % compact symbols for 1/2, etc.
\usepackage{microtype}      % microtypography
\usepackage[inline]{enumitem}
\usepackage{caption}
\usepackage{subcaption}
\usepackage{float}
% \usepackage{amsmath,amsthm}
% \usepackage{mathtools}
\usepackage{multirow}

% \usepackage{graphicx}
\usepackage{xspace}
% \usepackage[dvipsnames]{xcolor}
% \usepackage[unicode=true,
%  bookmarks=false,
%  breaklinks=false,
%  pdfborder={0 0 1},
%  backref=page]
%  {hyperref}
% \hypersetup{
%  colorlinks,linkcolor=red,
%  anchorcolor=blue,
%  citecolor=NavyBlue,
%  colorlinks = true
%  }
% \usepackage{backref}

%%%%%%%%%%%%%%%%%%%%%%% Paper specific %%%%%%%%%%%%%%%%%%%%%%%
\newcommand{\sm}{SM-2}
\newcommand\blfootnote[1]{%
  \begingroup
  \renewcommand\thefootnote{}\footnote{#1}%
  \addtocounter{footnote}{-1}%
  \endgroup
}
\newcommand{\para}[1]{\noindent{\textbf{#1}}}
\newcommand{\figref}[1]{Fig.~\ref{#1}}
\newcommand{\secref}[1]{\S\ref{#1}}
%%%%%%%%%%%%%%%%%%%%%%%%%%%%%%%%%%%%%%%%%%%%%%%%%%%%%%%%%%%%%%


\setlength{\dbltextfloatsep}{6pt}
\setlength{\dblfloatsep}{8pt}
\setlength{\floatsep}{2pt}
\setlength{\textfloatsep}{8pt}
\captionsetup[subfigure]{aboveskip=4pt,belowskip=2pt}
\captionsetup[table]{aboveskip=4pt,belowskip=2pt}
\captionsetup[figure]{aboveskip=4pt,belowskip=2pt}

\newcommand{\baselineNotation}{}
\newcommand{\mtl}{\texttt{MTL}\xspace}
\newcommand{\resn}{\texttt{RESNET}\xspace}
\newcommand{\tn}{\texttt{TripletNet}\xspace}


%%%%%%%%%%%%%%%%%%%%%%% Math %%%%%%%%%%%%%%%%%%%%%%%
\theoremstyle{plain}
% \newtheorem{assumption}{Assumption}
\newcommand{\g}{\ensuremath{g}}
\newcommand{\f}{\ensuremath{f}}
\newcommand{\inputspace}{\ensuremath{\mathbb{X}}}
\newcommand{\outputspace}{\ensuremath{\mathbb{Y}}}
%%%%%%%%%%%%%%%%%%%%%%%%%%%%%%%%%%%%%%%%%%%%%%%%%%%%%%%%%%%%%%

%%%%%%%%%%%%%%%%%%%%%%% Comments %%%%%%%%%%%%%%%%%%%%%%%
\newif\ifcomments

\commentstrue

\ifcomments
    \newcommand\chenhao[1]{\textcolor{magenta}{[CT: #1]}}
    \newcommand\shi[1]{\textcolor{brown}{[Shi: #1]}}
    \newcommand\harry[1]{\textcolor{gray}{[HT: #1]}}
    \newcommand\han[1]{\textcolor{violet}{[Han: #1]}}
    \newcommand\cc[1]{\textcolor{blue}{[Chacha: #1]}}
    \newcommand\yuxin[1]{\textcolor{red}{[Yuxin: #1]}}
\else
    \newcommand\chenhao[1]{}
    \newcommand\shi[1]{}
    \newcommand\harry[1]{}
    \newcommand\han[1]{}
    \newcommand\yuxin[1]{}
\fi
%%%%%%%%%%%%%%%%%%%%%%%%%%%%%%%%%%%%%%%%%%%%%%%%%%%%%%%%%%%%%%




% The \icmltitle you define below is probably too long as a header.
% Therefore, a short form for the running title is supplied here:
% \icmltitlerunning{Submission and Formatting Instructions for ICML 2022}

\begin{document}

\twocolumn[
% \icmltitle{Submission and Formatting Instructions for \\
%            International Conference on Machine Learning (ICML 2022)}

\icmltitle{Towards Effective Case-Based Decision Supports \\
            with Human-Compatible Representations}

% It is OKAY to include author information, even for blind
% submissions: the style file will automatically remove it for you
% unless you've provided the [accepted] option to the icml2022
% package.

% List of affiliations: The first argument should be a (short)
% identifier you will use later to specify author affiliations
% Academic affiliations should list Department, University, City, Region, Country
% Industry affiliations should list Company, City, Region, Country

% You can specify symbols, otherwise they are numbered in order.
% Ideally, you should not use this facility. Affiliations will be numbered
% in order of appearance and this is the preferred way.
\icmlsetsymbol{equal}{*}

\begin{icmlauthorlist}
\end{icmlauthorlist}

% You may provide any keywords that you
% find helpful for describing your paper; these are used to populate
% the "keywords" metadata in the PDF but will not be shown in the document
\icmlkeywords{Human-AI collaboration}

\vskip 0.3in
]

% this must go after the closing bracket ] following \twocolumn[ ...

% This command actually creates the footnote in the first column
% listing the affiliations and the copyright notice.
% The command takes one argument, which is text to display at the start of the footnote.
% The \icmlEqualContribution command is standard text for equal contribution.
% Remove it (just {}) if you do not need this facility.

%\printAffiliationsAndNotice{}  % leave blank if no need to mention equal contribution
% \printAffiliationsAndNotice{\icmlEqualContribution} % otherwise use the standard text.

%!TEX root = main.tex

\begin{abstract}
    \han{changed a few sentences here}
    Algorithmic case-based decision support provides examples to help human make sense of predicted labels and aid human in decision-making tasks. 
    % Despite the promising performance of supervised learning, 
    However, 
    representations learned by supervised models may not align well with human intuitions: what models consider as similar examples can be perceived as distinct by humans.
    % As a result, they have limited effectiveness in case-based decision support.
    In this work, we incorporate ideas from metric learning with supervised learning to examine the importance of alignment for effective decision support.
    In addition to instance-level labels, we use human-provided triplet judgments to learn human-compatible decision-focused representation.
    Using human subject experiments, we demonstrate that such representation is better aligned with human perception than representation solely optimized for classification.
    Human-compatible representations identify nearest neighbors that are perceived as more similar by humans and allow humans to make more accurate predictions.
\end{abstract}

%!TEX root = main.tex

\section{Introduction}

In recent years machine learning and AI has seen impressive development and drew great attention, from ML models surpassing human classification benchmarks to neural networks generating realistic text and even human faces. However, (and quite reasonably) ML models has shown less proficiency in learning or aligning with human values. In this work, we investigate human alignment in the form of human similarity judgements: given the question "Is \textit{a} more similar to \textit{b} than to \textit{c}?”, human provide feedback that can be interpreted as a constraint $dist(a, b) < dist(a, c)$. Learning such a perceptual distance metric or an embedding that respects the constraint can be useful in tasks that involve human-AI collaboration. 

The problem of learning human similarity judgment is not a novel one. There is abundant work on solving this problem, such as Stochastic Triplet Embedding (STE) \citep{2012tste}, Crowd Kernel Learning \citep{2011CKL} active multidimensional scaling \citep{2011aMDS}, and many more \citep{roads2020psiz,chen2019loe}. As ~\cite{chen2019loe} noted, all of these methods involve expensive computations and face scalability problems. This work does not aim to solve those issues or propose an algorithm surpassing the state-of-the-art, but instead fall back to the simplest (and perhaps more interpretable) matrix completion methods. From analyzing the performance of simpler methods we hope to find quantitative or qualitative patterns in human similarity judgment.

In this work we implement the singular value thresholding algorithm \citep{2008svt} as our matrix completion method. Given human similarity judgments in the form of triplet annotations, we experiment with different distance matrix constructions methods. We find that overall, matrix produced by SVT perform better than our baselines. We also experiment with varying number of data and find that...

% \input{formulation}
%!TEX root = main.tex
% \section{Metric Learning for Case-Based Decision Support}
\section{Case-Based Decision Support}
\label{sec:formulation}



Consider the problem of using a classification model $h: \inputspace \rightarrow \outputspace$ as decision support for humans.
Simply showing the predicted label from the model provides limited information and ``explanations'' are commonly hypothesized to improve human performance \citep{doshi2017towards}.
We focus on information presented in the form of examples from the training data, also known as case-based decision support \cite{kolodneer1991improving,begum2009case,liao2000case,angehrn1998case}.
Case-based decision support can have diverse use cases and goals.
Given a test example ($x$) and its predicted label ($\hat{y}$), two common use cases are:
\begin{itemize}[topsep=0pt, leftmargin=*, itemsep=0pt]
  \item Presenting the nearest neighbor of $x$ with label $\hat{y}$ as a justification of the predicted label. We refer to this scenario as {\em justification}.
  \item Presenting the nearest neighbor in each class without presenting $\hat{y}$. This approach makes a best-effort attempt to provide evidence for each class and leaves the final decision to human, without biasing human with the predicted label. We refer to this scenario as {\em decision support}.
\end{itemize}

\paragraph{Formulation.}
In this work, we formalize the problem of case-based decision support. 
The goal is to assist humans on a classification problem with groundtruth $\f: \inputspace \rightarrow \outputspace$.
We assume access to a representation model $g$, which takes an input $x$ and generates an $m$-dimensional representation $g(x) \in R^m$.
For each test instance $x$, an example selection policy $\pi$ chooses $k$ labeled examples from the training set $D^\text{train}$ and show them to the human (optionally along with the labels); the human then makes a prediction by choosing a label from $\outputspace$.
%% CT-0517: this is not essential?
% Without loss of generality, we will assume that the classification problem is binary from now on.
% In our definition, the main role of the model $g$ is to provide embeddings as inputs to the example selection policy, not making predictions autonomously, so we do not have to obtain representations from a classification model.
% That said, we use representation in a supervised classification model as our main baseline because it is the defacto choice in practice. 
% \chenhao{add citation}
As discussed in the two common use cases, example selection policy tends to be some form of nearest neighbor since it is important to be interpretable.
The focus of this work is thus on the effectiveness of $g$ for case-based decision support.

% \chenhao{not sure we need this paragraph}
% We refer to the embedding model $g$ (or \textit{model} for short) and the example selection policy $\pi$ (or \textit{policy} for short) collectively as the decision support system (or \textit{the system} for short).

Given a classification model $h$, the representation in justification and decision support is a byproduct derived from $h$, which is the last layer before the classification head.
We refer to this model as $e(h)$.\footnote{In general, we can use the representation in any layer, but in preliminary experiments, we find representation from the last layer is most effective.}
In justification, the example policy is $NN(x, e(h), D^\text{train}_{\hat{y}})$, where $D^\text{train}_{\hat{y}}$ refers to the subset of training data with label $\hat{y}$ and $NN$ finds the nearest neighbor of $x$ using representations from $e(h)$ among the subset of examples with label $\hat{y}$.
In decision support, the example policy is $\{NN(x, e(h), D^\text{train}_{y}), ~\forall y \in \outputspace \}$.

\han{removed misalignment paragraph here.}
% \paragraph{Misalignment with human similarity metric is detrimental.}
% To reason about the effectiveness of a representation model, we need to think about the goal of case-based decision support.
% Let us start with justification, which is a relatively easy case. 
% To justify a predicted label, the chosen example should ideally {\em appear similar} to the test image.
% Crucially, this similarity is perceived by humans and the example policy selects the nearest neighbor based on model representation.
% The gap between human representation and model representation (\figref{fig:model_vs_human}) leads to undesirable justification.

% Decision support, however, represents a more complicated scenario.
% We start by emphasizing that the goal is not simply to maximize human decision accuracy, because that goal can invite policies that may intentionally show examples that are far away to nudge or deceive human towards making a particular decision.
% Choosing nearest neighbors in each class is thus an attempt to present the most reliable evidence from the representation model so that humans can make their own decisions, hence preserving their agency.
% Therefore, the chosen nearest neighbors should be visually similar to the test instance by human perception, again highlighting the potential gap between model representation and human representation.
% Assuming that human follow the natural strategy by picking the presented instance that's most {\em similar} to the test instance and answering with the corresponding label, then ideally, nearest neighbors in each class retain key information useful for classification so that they can reveal the separation learned in the model.

% In both cases, aligning model representations with human similarity metric is crucial for case-based decision support; we refer to it as the \textit{metric alignment problem}.
% It is unlikely that we will get high alignment by default even when the model's classification accuracy is comparable to human.
% Models trained with supervised learning almost always exploit patterns in the training data that are
% \begin{enumerate*}[label=(\roman*)]
%   \item not robust to distribution shifts, and
%   \item counterintuitive or even unobservable for humans~\citep{ilyas2019adversarial,xiao2020noise}
% \end{enumerate*}.

% HAN: below are already commented in NeurIPS submission.
% When we are using more capable models to assist less capable human users, e.g., using an image classifier to assist radiology residents at breast cancer screening~\citep{wu2019deep,schaffter2020evaluation}, the human's incomplete knowledge about the task leads to a different kind of misalignment where the model relies on some legitimate feature that the human isn't aware of~\citep{kolodneer1991improving}.
% In both cases, the misalignment between the model and the human's similarity metric can lead to ineffective case-based decision support; 
% \shi{add another illustrative figure here?}

\paragraph{Combining metric learning on human triplets with supervised classification.}
We propose to address the metric alignment problem with additional supervision on the human similarity metric.
We collect data in the form of human similarity judgment triplets (or \textit{triplets} for short).
Each triplet is an ordered tuple: $(x^r, x^+, x^-)$, which indicates $x^+$ is judged by human as being closer to the reference $x^r$ than $x^-$~\citep{balntas2016learning}.

Given a triplet dataset $T$ and labeled classification dataset $D$, we use triplet margin loss~\citep{balntas2016learning} in conjunction with the cross-entropy loss, controlled by a hyperparameter $\lambda$:
\harry{I removed the equation numbers}
\begin{align*}
  &L = \lambda L_\mathrm{CrossEntropy(CE)} + (1-\lambda) L_\mathrm{TripletMargin(TM)} \label{eq:mtl_loss} \\
  &L_\mathrm{CE} = - \sum_{(x,y)\sim D} \log\left(p_\theta(y|x)\right) \\
  &L_\mathrm{TM} = \sum_{(x^r,x^+,x^-)\sim T} \max\left(d_\theta(x^r,x^+)-d(x^r,x^-)+\mu,0\right)
\end{align*}
% \begin{equation}
% \underbrace{\left[- \sum_{(x,y)\sim D} \log\left(p_\theta(y|x)\right)\right]}_\textrm{Cross-entropy loss} + (1-\lambda)\underbrace{\left[\sum_{(x^r,x^+,x^-)\sim T} \max\left(d_\theta(x^r,x^+)-d(x^r,x^-)+\mu,0\right)\right]}_\textrm{Triplet margin loss}
% \label{eq:mtl_loss}
% \end{equation}

where $d(\cdot,\cdot)$ is the similarity metric based on model representations, and $\mu$ is the margin hyperparameter; we use Euclidean distance and always set $\mu=1$.
We parameterize $\theta$ with a pretrained \resn~\citep{he2016deep}.
When $\lambda=1$ and the triplet margin loss is turned off, the model reduces to a finetuned \resn.
When $\lambda=0$ and the cross-entropy loss is turned off, the model reduces to \citet{balntas2016learning}'s mode; we call it \tn and treat it as a baseline.
We refer to our model as \mtl which stands for \textbf{M}ulti-\textbf{T}ask \textbf{L}earning.

% ; \mtl$_{0.2}$ refers to the \mtl model with $\lambda=0.2$
% Han: no longer using mtl_lambda in results.

% \chenhao{we should emphasize that we are learning representation and leave the selection of example policy to different work.}

% \subsection{Example selection policies} \yuxin{seems to be more appropriate to show up in the experiment section}
% 
% \para{AI model and baselines.}
% 
% \begin{itemize}[topsep=0pt,leftmargin=*,itemsep=-1pt]
%   \item \underline{\resn}~\citep{he2016deep}:
%   We use pretrained \resn model and finetuned on the downstream tasks.
%   \item \underline{\tn}: Another baseline we consider is directly learning human similarity measures with triplet margin loss~\citep{balntas2016learning}, as specified below:
%   \begin{equation}
%   L(a,p,n) = \max\{d(a_i,p_i)-d(a_i,n_i)+\text{margin},0\}
%   \end{equation}
%   We use \resn as the backbone of the \tn for fair comparison. Hence, $d(\cdot,\cdot)$ here is the euclidean distance between the \resn backbone embeddings.
%   \item \underline{\mtl}: In order to jointly learn the classification task and the human-compatible representations, we propose \mtl using the \resn backbone and the loss function as follows:
%   \begin{equation}
%   L = \lambda CE() + (1-\lambda) \text{triplet loss}
%   \end{equation}
%   % \item ResNet + MLP
%   % \item ResNet + ProtoNet
%   % \item ResNet + DWAC
%   % \item MTL with different lambda
%   % \item MTL with decision-consistent triplets
%   % \item MTLT with different lambda
%   % \item Triplet-only
%   % \item random
% \end{itemize}
% \harry{do we want to include a random baseline for selecting random decision suppport examples?}\yuxin{i think it's useful to keep random}



% The problem we are interested in this paper is case-based decision support with AI, where AI agent assists human to perform the end-task by retrieving cases that are helpful to human reasoning process. An overview of the setup is as follows: the AI agent learns to perform the underlying task $\g: \inputspace \rightarrow \outputspace$. Then we have a policy $\pi$ to retrieve a useful set of cases $D$ and show to human decision makers. Human ($h$) makes the decision given $D$ and its own intuition.


% \para{AI agent.} More formally, our AI agent consists of a predictor $g$ and a case retrieving policy $\pi$. The predictor $g$ learns the underlying task (mapping from $\inputspace$ to $\outputspace$) in a supervised manner. The retrieving policy $\pi$ is used to retrieve cases (instances) that will be useful to human decision makers.


% \para{Case-based decision support with AI.}


% \para{Human decision maker.} In our setting, we consider two important aspects of a human decision maker: human intuitions and human decision functions.
%
% First, we elaborate what do we mean by human intuitions and articulate our assumptions.
%
% \begin{assumption}[Form of human intuitions]
% Human intuition is defined as the similarity measure $K(\cdot,\cdot):\inputspace\times\inputspace\rightarrow \mathbb{R}_{\ge 0}$ that the human uses to determine the similarity between the two instances from the input space.
% \end{assumption}
%
% Next, we define human decision function $h$, where human decision is given by $h(K,D,x)$, i.e., humans make their decisions based on the specific instance $x$, their intuition (similarity measure $K$), the provided cases ($D$) from AI.
%
%
% \begin{assumption}[Form of human decision function]
% We assume humans make decisions based on nearest neighbors.
% \end{assumption}
%
%
% \para{Objective.}
% The system objective is to minimize human decision error given the AI retrieved cases $D$ and human intuition $K$.
%
% \begin{equation}
% L(g, h) = l(h(K,D,x), y)
% \end{equation}

% \input{alignment}
% \input{model}
%!TEX root = main.tex 
\section{Results}

%!TEX root = main.tex 

\begin{table}[t!]
    \centering
    \begin{tabular}{@{}llll@{}}
    \toprule
    method   & ID1   & IDR   & INC   \\ \midrule
    baseline & 0.377 & 0.435 & 0.317 \\
    SVT      & 0.588 & 0.581 & 0.454 \\ \bottomrule
    \end{tabular}
    \label{tab:main-results}
    \caption{Test triplet accuracies of distance matrices constructed by varying methods. All methods beat the baseline method.}
\end{table}

\begin{figure}[t]
    \center 
    \begin{subfigure}[b]{0.31\textwidth}
        \includegraphics[width=\textwidth]{figures/ID1.pdf}
    \end{subfigure}
    \begin{subfigure}[b]{0.31\textwidth}
        \includegraphics[width=\textwidth]{figures/IDR.pdf}
    \end{subfigure}
    \begin{subfigure}[b]{0.31\textwidth}
        \includegraphics[width=\textwidth]{figures/INC.pdf}
    \end{subfigure}
    \caption{Test triplet accuracies of distance matrices by varying number of triplets. All methods beat the baseline method even with few number of triplets.}
    \label{fig:main-results}
\end{figure}


Our baseline is simply test triplet accuracy using the distance matrix $D$ without matrix completion. We present our main results in Table 1. With SVT matrix completion, all our matrix construction methods outperform the baseline. 

We also present results on decreasing the number of triplets in ~\figref{fig:main-results} and show that our matrix constructed methods with SVT is robust to varying number of triplets.


% %!TEX root = main.tex

\section{Related Work}
\para{Connection to machine teaching}
Explanation-based teaching captures a broad class of machine teaching problems where labels provided by the teacher are coupled with additional information (such as highlighting regions or features on an image, supplementing the teaching instruction with contrastive examples \citep{wang2021teaching}). Explanations were shown to be effective in guiding a human learner to improve classification performance by steering the learner's attention \citep{grant2003,roads2016,chen2018near,macaodha18teaching}.
We note that the case-based decision support considered in this paper is intrinsically connected to explanation-based machine teaching in that both problems seek a set of (labeled) examples as instructions to improve human decision making, with two fundamental differences that span both problem setup and objective: (a) for case-based decision support, we assume human is the decision maker and only consult the model as an external aid when making predictions; the model always presents, even at the prediction phase. In contrast, classical machine teaching aims to distill the model fully to the human at the training phase, and human will make their decisions without the model at the prediction phase. (b) Due to such difference in problem setups, the best performance in case-based decision support is natually constrained and grounded by the human's (intial) model---as a \textit{nonparametric} decision model (e.d. 1-NN) conditioning on the provided decision support---and the focus is on the joint design of the representation and the example selection policy. % with extra consideration in human's trust of the model.
While most machine teaching work consider a fixed representation of (human) learners, it often assumes a \textit{parametrized} decision model of the learner, and the focus is on the design of an example selection policy that steers the learner to adopt the best parameterization.


\para{Ordinal embedding}
The ordinal embedding problem \citep{ghosh2019landmark,van2012stochastic,kleindessner2017kernel,kleindessner2014uniqueness,terada2014local,park2015preference} aim to find low-dimensional representations that respect ordinal feedback. Currently there exist several techniques for ordinal embedding. Generalized Non-metric Multidimensional Scaling (GNMDS) \citep{agarwal2007generalized} takes a max-margin approach by minimizing hinge loss. Stochastic Triplet Embedding (STE) \citep{van2012stochastic} assumes the Bradley-Terry-Luce (BTL) noise model \citep{bradley1952rank,luce1959individual} and minimizes logistic loss. The Crowd Kernel \citep{tamuz2011adaptively} and t-STE \citep{van2012stochastic} propose alternative non-convex loss measures based on probabilistic generative models. These results are primarily empirical and focus on minimizing prediction error on unobserved triplets. In principle, one can plugin these approaches to the \mtl model as alternatives to the triplet margin loss \ref{eq:mtl_loss}.
% \yuxin{can someone look into this paper \citep{nadagouda2022active} and include it as a reference?}
% >>>>>>> 46bc1c2482fcbd751baa2628e7249d54c9268b2b
%rely on expensive gradient or projection computations and are unsuitable for large datasets. The results in these papers are primarily empirical and focus on minimizing prediction error on unobserved triplets.

%!TEX root = main.tex

\section{Conclusion}
\label{sec:conclusion}

Human decision making can be assisted by showing examples that are similar to the current problem that they are working on.
As we identify in this paper, the key to providing effective case-based support with a model is the alignment between the model and the human in terms of similarity metrics: two examples that appear similar to the model should also appear similar to the human.
But models trained to do classification do not automatically produce representations that satisfy this property.
To address this issue, we propose a multi-task learning method (\mtl) to combine two sources of supervision: labeled examples for classification, and triplets of human similarity judgments.
With synthetic experiments and user studies, we validate that \mtl
\begin{enumerate*}[label=(\roman*)]
  \item consistently get the best of both worlds in terms of classification accuracy and triplet accuracy,
  % achieves a classification accuracy comparable to training only on the classification examples, a triplet accuracy comparable to only training to approximate human similarity metrics, and does it consistently for easy and hard (in terms of alignment score) setting, meaning that it can reliably get the best of both worlds.
  \item selects visually more similar examples in head-to-head comparisons,
  \item and provides better decision support% with various example selection policies.
\end{enumerate*}.

\para{Limitations.} A list of limitations that we hope to address in future work:
\begin{enumerate*}[label=(\roman*)]
\item We only have one model architecture and one task in human study.
\item Our experiments are focused on simple classification tasks for the model and are not representative of high-stake domains.
% \item Our results show that triplet judgments are an effective way to incorporate human intuitions, however, it remains an open question whether this approach is efficient
% \item The example selection policies are heuristic-based.
% \item We do not have a good way to control for the amount training data for baselines.
% \item We do not have a good explanation for why filtering improves decision support.
\end{enumerate*}


\clearpage

\bibliography{main}
\bibliographystyle{icml2022}

\clearpage

%%%%%%%%%%%%%%%%%%%%%%%%%%%%%%%%%%%%%%%%%%%%%%%%%%%%%%%%%%%%%%%%%%%%%%%%%%%%%%%
%%%%%%%%%%%%%%%%%%%%%%%%%%%%%%%%%%%%%%%%%%%%%%%%%%%%%%%%%%%%%%%%%%%%%%%%%%%%%%%
% APPENDIX
%%%%%%%%%%%%%%%%%%%%%%%%%%%%%%%%%%%%%%%%%%%%%%%%%%%%%%%%%%%%%%%%%%%%%%%%%%%%%%%
%%%%%%%%%%%%%%%%%%%%%%%%%%%%%%%%%%%%%%%%%%%%%%%%%%%%%%%%%%%%%%%%%%%%%%%%%%%%%%%
\newpage
\appendix
\onecolumn
%!TEX root = main.tex 
% \section{You \emph{can} have an appendix here.}



%!TEX root = main.tex 

% \clearpage
\section{Implementation detail}
% \harry{move the deleted stuff in the main paper to here}
\label{sec:implementation}

The architecture of our model is presented in \ref{fig:architecture}. 
We first encode image inputs using a Convolutional Neural Network (CNN), in this case a pretrained \resn-18~\citep{he2016deep}, and then project the output into an high-dimension representation space with a projection head made of multi-layer perceptron (MLP). 
In our experiments we use one non-linear layer to project the output of the CNN into our representation space. 
For classifcation task we add an MLP classifier head. 
We also use one non-linear layer with softmax activation. 
For triplet prediction, we will re-index the representations with the current triplet batch and calculate prediction or loss.
We use the PyTorch framework \cite{pytorch} and the PyTorch Lightning framework \cite{pytorch-lightning} for implementation.
Hyperparameters will be reported in \ref{sec:supp_human} for models in the human experiments and in \ref{sec:supp_syn} for models in the synthetic experiments.

\begin{figure}[t]
    \centering
    \includegraphics[width=\textwidth]{figures/general/architecture.pdf}
    \caption{Architecture of the \mtl model.}
    \label{fig:architecture}
\end{figure}

% \subsection{Hyperparameters}
% \cc{add model hyperparameter}
% Han: pushed to VW and BM sections.

% \chenhao{add model structure}

% \han{do we talk about random seed?}
% \chenhao{probably no}


\subsection{Computation resources}
We use a computing cluster at our institution. We train our models on nodes with different GPUs including Nvidia GeForce RTX2080Ti, Nvidia GeForce RTX3090, Nvidia Quadro RTX 8000, and Nvidia A40. All models are trained on one allocated node with one GPU access.


% \subsection{Reproducibility}
% Our source code of the implementation is included in this url: 
% We submit a preliminary version of the code and data.
% We will release more details upon acceptance.
% \han{To be zipped and upload: MTL model, trainer, evaluation script}.
% \harry{We provde soure code if acceptance?}

% Data? \han{Shall we publish the data as well?}

% Upon acceptance?


%!TEX root = main.tex 

% \clearpage
\section{Human subject study}
\label{sec:supp_human}

\subsection{Dataset}
Our BM dataset include four species of butterflies and moths including: Peacock Butterfly, Ringlet Butterfly, Caterpiller Moth, and Tiger Moth. An example of each species is shown in Fig \ref{fig:bm-species}.

\begin{figure}[t]
    \centering
    \begin{subfigure}{0.24\textwidth}
      \includegraphics[width=\textwidth]{figures/bm/species/ringlet.jpg}
      \caption{Ringlet Butterfly}
    \end{subfigure}
    \begin{subfigure}{0.24\textwidth}
        \includegraphics[width=\textwidth]{figures/bm/species/peacock.jpg}
        \caption{Peacock Butterfly}
    \end{subfigure}
    \begin{subfigure}{0.24\textwidth}
        \includegraphics[width=\textwidth]{figures/bm/species/caterpillar.jpg}
        \caption{Caterpiller Moth}
    \end{subfigure}
    \begin{subfigure}{0.24\textwidth}
        \includegraphics[width=\textwidth]{figures/bm/species/tiger.jpg}
        \caption{Tiger Moth}
    \end{subfigure}
    \caption{An example of each species in the BM dataset.}
    \label{fig:bm-species}
\end{figure}

\subsection{Hyperparameters}
We use different controlling strength between classification and human judgment prediction, including $\lambda$s at 0.2, 0.5, and 0.8.
We use the Adam optimizer \cite{kingma2014adam} with learning rate $1e-4$.
Our training batch size is $120$ for triplet prediction, and $30$ for classification.
% \chenhao{how do you stop? choose the number of epoches?}
All models are trained for 50 epoches. The checkpoint with the lowest validation total loss in each run is selected for evaluations and applications.

\subsection{Classification and Triplet Accuracy}
%!TEX root = ../supp_main.tex

\begin{table}[t]
    \centering
    \begin{tabular}{|cc|cc|cc|}
        \hline
        \multicolumn{2}{|c|}{}                                            & \multicolumn{2}{c|}{Filtered}                            & \multicolumn{2}{c|}{Unfiltered}            \\ \hline
        \multicolumn{1}{|c|}{Dim.}                 & Model                & \multicolumn{1}{c|}{Clf Acc}          & Triplet Acc      & \multicolumn{1}{c|}{Clf Acc} & Triplet Acc \\ \hline
        \multicolumn{1}{|c|}{\multirow{3}{*}{50}}  & \resn & \multicolumn{1}{c|}{(see unfiltered)} & (see unfiltered) & \multicolumn{1}{c|}{0.975}   & 0.610       \\ \cline{2-6} 
        \multicolumn{1}{|c|}{}                     & \tn   & \multicolumn{1}{c|}{N/A}              & 0.721            & \multicolumn{1}{c|}{N/A}     & 0.759       \\ \cline{2-6} 
        \multicolumn{1}{|c|}{}                     & \mtl  & \multicolumn{1}{c|}{0.975}            & 0.707            & \multicolumn{1}{c|}{0.975}   & 0.762       \\ \hline
        \multicolumn{1}{|c|}{\multirow{3}{*}{512}} & \resn & \multicolumn{1}{c|}{(see unfiltered)} & (see unfiltered) & \multicolumn{1}{c|}{0.975}   & 0.631       \\ \cline{2-6} 
        \multicolumn{1}{|c|}{}                     & \tn   & \multicolumn{1}{c|}{N/A}              & 0.732            & \multicolumn{1}{c|}{N/A}     & 0.748       \\ \cline{2-6} 
        \multicolumn{1}{|c|}{}                     & \mtl  & \multicolumn{1}{c|}{0.975}            & 0.709            & \multicolumn{1}{c|}{1.000}   & 0.741       \\ \hline
    \end{tabular}
    \caption{Classification and triplet accuracy of BM models.
    % \chenhao{drop TN} \han{but we need to show TN triplet acc right? We used TN with highest triplet acc as the synthetic agent.}
    }
    \label{tab:bm-models}
\end{table}
We present the test-time classification and triplet accuracy of our models in Table \ref{tab:bm-models}. Both \resn and \mtl achieve above 97.5\% classification accuracy. \mtl in the 512-dimension unfiltered setting achieve 100.0\% classification accuracy. Both \tn and \mtl achieve above 70.7\% triplet accuracy. Both \tn model and \mtl achieve the highest triplet accuracy in the 50-dimension unfiltered setting with triplet accuracy at 75.9\% and 76.2\% respectively.

We also evaluate the pretrained LPIPS metric \cite{zhang2018perceptual} on our triplet test set as baselines for learning perceptual similarity.
Results with AlexNet backbone and VGG backbone are at 54.5\% and 55.0\% triplet accuracy respectively, suggesting that \tn and \mtl provides much better triplet accuracy in this task compared to a generic model.


\subsection{Subject Evaluations}
We include survey questions in the end of the decision support tasks. 
The two required questions are: 
1) accuracy belief: ``How many questions do you think you have answered correctly?'' 
2) support usefulness: ``Do you agree that the reference images are helpful when you decide the class of the test image?'' This is measured in a 5-point Likert scale.
Results are reported in Fig \ref{fig:subjective}. 

Subjective beliefs of accuracy display a similar trend as that in the objective measure of accuracy. \mtl NINO and \mtl NIFO are significantly better than RIRO, \resn NINO, and \resn NIFO. However, the differences among RIRO, \resn NINO and \resn NIFO are no longer significant.

Subjective beliefs of decision support usefulness display the same trend as that in the subjective measure of accuracy beliefs. \mtl NINO and \mtl NIFO are believed to be more useful than other decision supports, among which there is no significant difference.

\begin{figure}[t]
    \centering
    \begin{subfigure}[t]{0.48\textwidth}
        \includegraphics[width=\textwidth]{figures/bm/prolific/decision-acc_belief.pdf}
        \caption{Subjective evaluation of participants' beliefs on their accuracy.}
    \end{subfigure}
    \begin{subfigure}[t]{0.48\textwidth}
        \includegraphics[width=\textwidth]{figures/bm/prolific/decision-usefulness.pdf}
        \caption{Subjective evaluation of participants' beliefs on decision support usefulness.}
    \end{subfigure}
    \caption{Subjective evaluation in the decision support task.}
    \label{fig:subjective}
\end{figure}


% \clearpage
\subsection{Interface}
We present the screenshots of our interface in this section. 
Our interface consists of four stages. 
Participants will see the consent page at the beginning, as shown in Fig \ref{fig:interface_consent}. 
After consent page, participants will see task specific instructions, as shown in Fig \ref{fig:interface_prolific}. 
After entering the task, partipants will see the questions, as shown in Fig \ref{fig:interface_questions}. 
We also include two attention check questions in all studies to check whether participants are paying attention to the questions. 
Following suggestions on Prolific, we design the attention check with explicit instructions, as shown in Fig \ref{fig:interface_attention}.
After finishing all questions, participants will reach the end page and return to Prolific, as shown in Fig \ref{fig:interface_end}. 
Our study is reviewed by the Internal Review Board (IRB) at our institution with study number that we will release upon acceptance to preserve anonymity.
% IRB22-0388. 
% \han{Will we be identified through the IRB number?}


\begin{figure}
    \centering
    \includegraphics[width=\textwidth]{figures/interface/consent-annotate.pdf}
    \caption{The consent form page on our interface.}
    \label{fig:interface_consent}
\end{figure}

\begin{figure}
    \centering
    \begin{subfigure}[t]{0.48\textwidth}
        \includegraphics[width=\textwidth]{figures/interface/prolific-annotate.pdf}
        \caption{The annotation and head-to-head comparision task instructions.}
    \end{subfigure}
    \begin{subfigure}[t]{0.48\textwidth}
        \includegraphics[width=\textwidth]{figures/interface/prolific-decision.pdf}
        \caption{The decision support task instructions.}
    \end{subfigure}
    \caption{The task-specific instruction page on our interface.}
    \label{fig:interface_prolific}
\end{figure}

\begin{figure}
    \centering
    \begin{subfigure}[t]{0.48\textwidth}
        \includegraphics[width=\textwidth]{figures/interface/annotate.pdf}
        \caption{The annotation and head-to-head comparision task questions.}
    \end{subfigure}
    \begin{subfigure}[t]{0.48\textwidth}
        \includegraphics[width=\textwidth]{figures/interface/decision.pdf}
        \caption{The decision support task questions.}
    \end{subfigure}
    \caption{The task-specific questions on our interface.}
    \label{fig:interface_questions}
\end{figure}

\begin{figure}
    \centering
    \begin{subfigure}[t]{0.48\textwidth}
        \includegraphics[width=\textwidth]{figures/interface/attention-annotate.pdf}
        \caption{The annotation and head-to-head comparision task attention check questions.}
    \end{subfigure}
    \begin{subfigure}[t]{0.48\textwidth}
        \includegraphics[width=\textwidth]{figures/interface/attention-decision.pdf}
        \caption{The decision support task attention check questions.}
    \end{subfigure}
    \caption{The task-specific attention check questions on our interface.}
    \label{fig:interface_attention}
\end{figure}

\begin{figure}
    \centering
    \includegraphics[width=\textwidth]{figures/interface/survey-decision.pdf}
    \caption{The survey page of the decision support task on our interface.}
    \label{fig:interface_survey_decision}
\end{figure}

\begin{figure}
    \centering
    \begin{subfigure}[t]{0.48\textwidth}
        \includegraphics[width=\textwidth]{figures/interface/end-annotate.pdf}
        \caption{The annotation and head-to-head comparision task end page.}
    \end{subfigure}
    \begin{subfigure}[t]{0.48\textwidth}
        \includegraphics[width=\textwidth]{figures/interface/end-decision.pdf}
        \caption{The decision support task end page.\\ }
    \end{subfigure}
    \caption{The task-specific end page on our interface.}
    \label{fig:interface_end}
\end{figure}



\subsection{Crowdsourcing}
We recruit our participants on a crowdsourcing platform: Prolific (www.prolific.co) [April-May 2022].
We conduct three total studies: an annotation study, a decision support study, and a head-to-head comparison study.
We use the default standard sampling on Prolific for participant recruitment.
Eligible participants are limited to those reside in United States.
Participants are not allowed to attempt the same study more than once.

\para{Triplet annotation study}
We recruit 90 participants in total. We conduct a pilot study with 7 participants to test the interface, and recruit 83 participants for the actual collection of annotations. 3 participants fail the attention check questions and their responses are excluded in the results. We spend in total \$76.01 with an average pay at \$10.63 per hour. The median time taken to complete the study is 3'22''.

% ap1 12 9.34, ab1 14.55 10.67, ab2 10.97 24, ab3 9.4 32.
% median time 202.199s

\para{Decision support study}
We recruit 161 participants in total. 3 participants fail the attention check questions and their responses are excluded in the results. We take the first 30 responses in each conditon to compile the results. We spend in total \$126.40 with an average pay at \$9.32 per hour. The median time taken to complete the study is 3'53''.

% median time 232.611s


\para{Head-to-head comparison study}
We recruit 31 participants in total, where 1 participant fail the attention check questions and their responses are excluded in the results. We spend in total \$24.00 with an average pay at \$9.40 per hour. The median time taken to complete the study is 3'43''.

% median time 223.385s


%!TEX root = main.tex 

% \clearpage
\section{Synthetic experiments}
\label{sec:supp_syn}

In addition to BM, we also experiment with synthetic datasets as we can tune more variables and better understand the strengths and limitations of our method. Using simulated human similarity metrics, we control and vary the level of disagreement between the classification groundtruth and the human's knowledge.

\subsection{Synthetic dataset and simulated humans}

\begin{figure}[t]
  \centering
  \includegraphics[width=\textwidth]{figures/wv/wv.pdf}
   \caption{VW dataset. (a) shows the non-linear decision boundary of the dataset determined by two features: the head and the body size of the fictional insects. The Weevil has a mid-sized body and mid-sized head, while the Vespula does not. Tail length and texture are two non-informative features.}
   \label{fig:vw}
  \end{figure}

% \begin{table}[t]
%     \centering
%     \begin{subfigure}[b]{0.49\textwidth}
%     \includegraphics[width=\textwidth]{figures/supp/weights.png}
%     \end{subfigure}
%     \caption{Histogram of alignments generated by searching informative weights in powers of 2.}
%     \label{fig:align-hist}
% \end{table}

We use the synthetic dataset ``Vespula vs Weevil'' (VW) from \citet{chen2018near}.
It is a binary image classification dataset of two fictional species of insects.
Each example contains four features, two of them---head and body size---are predictive of the label, the other two---tail length and texture---are completely non-predictive.
We generate 2000 images and randomly split the dataset into training, validation, and testing sets in a 60\%:20\%:20\% ratio.
The labels are determined by various synthetic decision boundaries, such as the one shown in \figref{fig:vw}a.
We report the results for \figref{fig:vw}a, a double-square decision boundary, in section C.2 and results for a linear decision boundary data in section C.3.


To generate triplets data, we simulate human visual similarity metrics by adding weights to each feature in Euclidean distance computation.
By changing the weight on each feature, we can control the level of disagreement between simulated human and the groundtruth.
All procedures that involve humans (e.g., triplet data collection and evaluation) are replaced by the simulated human in this set of experiments.

To quantify the disagreement, we use 1-NN classification accuracy following the simulated human similarity metric; we refer to it as the alignment score.
The alignment score ranges from 50\% (setting the informative features' weights to 0 and distractor weights to 1) to 100\% (setting all weights to 1).
We generate alignment scores by searching through weight combinations of the simulated human visual similarity metrics. We search the weights in powers of 2, from 0 to $2^{10}$, producing a sparse distribution of alignments.
%  (~\figref{fig:align-hist}). Increasing search range to powers of 10 produces smoother distribution, but the weights are also more extreme and unrealistic. We note that the alignment distribution may vary across different datasets. 
 In our experiments we choose weights and alignments to be as representative to the distribution as possible. In each alignment setting, we generate 40,000 triplets using the simulated humans similarity metric.



\subsection{Double-square decision boundary experiment results}


\subsubsection{Hyperparameters} 
We train all models with a large emebdding dimension of 512 and a small emebdding dimension of 50 and observe that, in contrast to the experiments on BM, a 512-dimension embeddings is preferable based on most metrics and models. 
We also train \mtl on a filtered vs. unfiltered triplets as well as with different values $\lambda$ at 0.2, 0.5, and 0.8. 
For our main results, we report the performance with $\lambda=0.5$ and filtered triplets.
We will discuss the effect of filtering later in this section.  


We use the Adam optimizer \cite{kingma2014adam} with learning rate $1e-4$.
We use a training batch size of $40$ for triplet prediction, and $30$ for classification.


\subsubsection{Main results}

%!TEX root = ../../supp_main.tex

\begin{table}[h]
    \centering
    \begin{tabular}{@{}lll@{}}
    \toprule
    model  & \textbf{classification accuracy} & \textbf{triplet accuracy} \\ \midrule
    \resn   & 0.998 $\pm$ 0.003                & 0.673 $\pm$ 0.014         \\
    \mtl $\lambda=0.8$ & 0.998 $\pm$ 0.032                & 0.970 $\pm$ 0.024         \\
    \mtl $\lambda=0.5$ & 0.995 $\pm$ 0                    & 0.972 $\pm$ 0.004         \\
    \mtl $\lambda=0.2$ & 0.996 $\pm$ 0.016                & 0.973 $\pm$ 0.039         \\
    \tn     & N/A                              & 0.973 $\pm$ 0.016         \\ \bottomrule
    \end{tabular}
    
    \caption{classification and triplet accuracy of \mtl with different $\lambda$. \tn has no classfication head and no classification accurayc.}
    \label{tab:wv-square-clf-trip}
\end{table}
%!TEX root = ../supp_main.tex

\begin{table}[t]
  \small
  \centering
  \resizebox{\textwidth}{!}{
  \begin{tabular}{@{}lrrrrrr@{}}
  \toprule
  Alignments   & 50\%   & 80\%   & 83\%  & 92\%  & 92.5\% & 100\%     \\ \midrule
  Weights  &  [0,0,1,1]  & [1,0,1,1] &  [0,1,1,1] &  [1,256,256,256] &  [256,1,256,256] &  [1,1,1,1]  \\ \midrule

  \multicolumn{7}{c}{\textbf{NI-H2H}} \\ \midrule
  \mtl vs. \resn     & 0.917 $\pm$ 0.064 & 0.914 $\pm$ 0.007 & 0.903 $\pm$ 0.016 & 0.880 $\pm$ 0.022 & 0.872 $\pm$ 0.020 & 0.808 $\pm$ 0.017\\ \midrule

  \multicolumn{7}{c}{\textbf{NO-H2H}} \\ \midrule
  \mtl vs. \resn     & 0.916 $\pm$ 0.093 & 0.968 $\pm$ 0.011 & 0.946 $\pm$ 0.009 & 0.958 $\pm$ 0.031 & 0.962 $\pm$ 0.008 & 0.970 $\pm$ 0.008\\ \midrule

  \multicolumn{7}{c}{\textbf{NINO Decision Support}}       \\ \midrule
  \resn     & 0.753 $\pm$ 0.056 & 0.899 $\pm$ 0.025 & 0.896 $\pm $0.044 & 0.897 $\pm$ 0.045 & 0.901 $\pm$ 0.025 & 0.929 $\pm$ 0.028 \\
  \tn       & 0.568 $\pm$ 0.049 & 0.775 $\pm$ 0.084 & 0.807 $\pm$ 0.038 & 0.868 $\pm$ 0.012 & 0.877 $\pm$ 0.025 & \textbf{1.000} $\pm$ 0.000\\ 
  \mtl      & \textbf{0.759} $\pm$ 0.080 & \textbf{0.901} $\pm$ 0.016& \textbf{0.928} $\pm$ 0.099 & \textbf{0.949} $\pm$ 0.034 & \textbf{0.955} $\pm$ 0.027 & \textbf{1.000} $\pm$ 0.00 \\ \midrule

  \multicolumn{7}{c}{\textbf{NIFO Decision Support}}       \\ \midrule
  \resn     & 0.753 $\pm$ 0.056 & 0.899 $\pm$ 0.025 & 0.896 $\pm $0.044 & 0.897 $\pm$ 0.045 & 0.901 $\pm$ 0.025 & 0.929 $\pm$ 0.028 \\
  \tn       & 0.568 $\pm$ 0.049 & 0.775 $\pm$ 0.084 & 0.807 $\pm$ 0.038 & 0.868 $\pm$ 0.012 & 0.877 $\pm$ 0.025 & \textbf{1.000} $\pm$ 0.000\\ 
  \mtl      & \textbf{1.000} $\pm$ 0.000 & \textbf{1.000} $\pm$ 0.000 & \textbf{1.000} $\pm$ 0.000 & \textbf{1.000} $\pm$ 0.000 & \textbf{1.000} $\pm$ 0.000 & \textbf{1.000} $\pm$ 0.000 \\ \midrule
  \end{tabular}}
  \caption{Experiment results on VW double-square decision boundary . Models use 512-dimension embeddings; \mtl uses $\lambda=0.5$ and filtered triplets.}
  \label{tab:table1-ci}
  \end{table}

\begin{figure}[h!]
      \centering
      \begin{subfigure}[b]{0.32\textwidth}
        \includegraphics[width=\textwidth]{figures/filtered_NINO_err.pdf}
        \caption{NINO decision support}
        \end{subfigure}
      \begin{subfigure}[b]{0.32\textwidth}
        \includegraphics[width=\textwidth]{figures/supp/wv_square_NI-h2h_filter.pdf}
        \caption{NI-H2H with \resn}
      \end{subfigure}
        \begin{subfigure}[b]{0.32\textwidth}
        \includegraphics[width=\textwidth]{figures/supp/wv_square_NO-h2h_filter.pdf}
        \caption{NO-H2H with \resn}
      \end{subfigure}
        \caption{Decision support and H2H performance on VW double-square decision boundary data, comparing \mtl with filtered and unfiltered triplets.}
        \label{fig:square-filter}
\end{figure}
  
\begin{figure}[t]
      \centering
      \begin{subfigure}[b]{0.32\textwidth}
        \includegraphics[width=\textwidth]{figures/noise_h2h_v1.pdf}
        \end{subfigure}
        \begin{subfigure}[b]{0.32\textwidth}
        \includegraphics[width=\textwidth]{figures/noise_NINO_v1.pdf}
        \end{subfigure}
        \begin{subfigure}[b]{0.32\textwidth}
        \includegraphics[width=\textwidth]{figures/noise_NIFO.pdf}
        \end{subfigure}
        \caption{Results on VW double-square decision boundary data with varying number of triplets. \mtl uses filtered triplets}
        \label{fig:noise}
      \end{figure}

\begin{figure}[h!]
  \centering
  \begin{subfigure}[b]{0.32\textwidth}
    \includegraphics[width=\textwidth]{figures/supp/wv_square_noise_unfiltered_h2h.pdf}
    \end{subfigure}
    \begin{subfigure}[b]{0.32\textwidth}
      \includegraphics[width=\textwidth]{figures/supp/wv_square_noise_unfiltered_NINO.pdf}
    \end{subfigure}
    \begin{subfigure}[b]{0.32\textwidth}
      \includegraphics[width=\textwidth]{figures/supp/wv_square_noise_unfiltered_NIFO.pdf}
    \end{subfigure}
    \caption{Results on VW double-square decision boundary data with varying number of triplets. \mtl uses unfiltered triplets}
    \label{fig:noise-unfiltered}
\end{figure}

\begin{figure}[h!]
  \begin{subfigure}[b]{0.32\textwidth}
          \includegraphics[width=\textwidth]{figures/num_h2h_v1.pdf}
          \end{subfigure}
          \begin{subfigure}[b]{0.32\textwidth}
          \includegraphics[width=\textwidth]{figures/num_NINO_v1.pdf}
          \end{subfigure}
          \begin{subfigure}[b]{0.32\textwidth}
          \includegraphics[width=\textwidth]{figures/num_NIFO_v1.pdf}
          \end{subfigure}
          \caption{Results on VW double-square decision boundary data with varying number of triplets. \mtl uses unfiltered triplets}
          \label{fig:vary-num}
    \end{figure}

%!TEX root = ../supp_main.tex

\begin{table}[h]
  \small
  \centering
  \begin{tabular}{@{}lrrrrrr@{}}
  \toprule
  Alignments   & 50\%   & 80\%   & 83\%  & 92\%  & 92.5\% & 100\%     \\ \midrule
  Weights  &  [0,0,1,1]  & [1,0,1,1] &  [0,1,1,1] &  [1,256,256,256] &  [256,1,256,256] &  [1,1,1,1]  \\ \midrule

  \multicolumn{7}{c}{\textbf{NI-H2H}} \\ \midrule
  \mtl vs. \resn     & 0.920 $\pm$ 0.005 & 0.890 $\pm$ 0.032 & 0.906 $\pm$ 0.053 & 0.895 $\pm$ 0.016 & 0.862 $\pm$ 0.254 & 0.832 $\pm$ 0.058\\ \midrule

  \multicolumn{7}{c}{\textbf{NO-H2H}} \\ \midrule
  \mtl vs. \resn     & 0.901 $\pm$ 0.439 & 0.948 $\pm$ 0.095 & 0.970 $\pm$ 0.019 & 0.972 $\pm$ 0.095 & 0.933 $\pm$ 0.154 & 0.981 $\pm$ 0.040\\ \midrule

  \multicolumn{7}{c}{\textbf{NINO Decision Support}}       \\ \midrule
  \resn     & \textbf{0.753} $\pm$ 0.056 & 0.899 $\pm$ 0.025 & 0.896 $\pm $0.044 & 0.897 $\pm$ 0.045 & 0.901 $\pm$ 0.025 & 0.929 $\pm$ 0.028 \\
  \tn       & 0.568 $\pm$ 0.049 & 0.775 $\pm$ 0.084 & 0.807 $\pm$ 0.038 & 0.868 $\pm$ 0.012 & 0.877 $\pm$ 0.025 & \textbf{1.000} $\pm$ 0.000\\ 
  \mtl      & 0.740 $\pm$ 0.540 & \textbf{0.925} $\pm$ 0.127 & \textbf{0.933} $\pm$ 0.064 & \textbf{0.935} $\pm$ 0.000 & \textbf{0.945} $\pm$ 0.349 & \textbf{1.000} $\pm$ 0.000 \\ \midrule

  \multicolumn{7}{c}{\textbf{NIFO Decision Support}}       \\ \midrule
  \resn     & 0.753 $\pm$ 0.056 & 0.899 $\pm$ 0.025 & 0.896 $\pm $0.044 & 0.897 $\pm$ 0.045 & 0.901 $\pm$ 0.025 & 0.929 $\pm$ 0.028 \\
  \tn       & 0.568 $\pm$ 0.049 & 0.775 $\pm$ 0.084 & 0.807 $\pm$ 0.038 & 0.868 $\pm$ 0.012 & 0.877 $\pm$ 0.025 & \textbf{1.000} $\pm$ 0.000\\ 
  \mtl      & \textbf{0.996} $\pm$ 0.016 & \textbf{0.995} $\pm$ 0.000 & \textbf{0.998} $\pm$ 0.000 & \textbf{0.996} $\pm$ 0.016 & \textbf{0.995} $\pm$ 0.000 & 0.995 $\pm$ 0.032 \\ \midrule

  \end{tabular}
  \caption{Experiment results on VW with \mtl using $\lambda=0.2$.}
  \label{tab:wv_square_filtered_l=0.2}
  \end{table}

% 
\begin{table}[t]
  \small
  \centering
    \resizebox{\textwidth}{!}{
  \begin{tabular}{@{}lrrrrrr@{}}
  \toprule
  Alignments   & 50\%   & 80\%   & 83\%  & 92\%  & 92.5\% & 100\%     \\ \midrule
  Weights  &  [0,0,1,1]  & [1,0,1,1] &  [0,1,1,1] &  [1,256,256,256] &  [256,1,256,256] &  [1,1,1,1]  \\ \midrule

  \multicolumn{7}{c}{\textbf{NI-H2H}} \\ \midrule
  \mtl vs. \resn     & 0.916 $\pm$ 0.082 & 0.869 $\pm$ 0.217 & 0.891 $\pm$ 0.029 & 0.879 $\pm$ 0.066 & 0.853 $\pm$ 0.164 & 0.828 $\pm$ 0.138\\ \midrule

  \multicolumn{7}{c}{\textbf{NO-H2H}} \\ \midrule
  \mtl vs. \resn     & 0.902 $\pm$ 0.193 & 0.944 $\pm$ 0.093 & 0.959 $\pm$ 0.005 & 0.956 $\pm$ 0.090 & 0.942 $\pm$ 0.026 & 0.969 $\pm$ 0.034\\ \midrule

  \multicolumn{7}{c}{\textbf{NINO Decision Support}}       \\ \midrule
  \resn     & \textbf{0.753} $\pm$ 0.056 & \textbf{0.899} $\pm$ 0.025 & 0.896 $\pm $0.044 & 0.897 $\pm$ 0.045 & 0.901 $\pm$ 0.025 & 0.929 $\pm$ 0.028 \\
  \tn       & 0.568 $\pm$ 0.049 & 0.775 $\pm$ 0.084 & 0.807 $\pm$ 0.038 & 0.868 $\pm$ 0.012 & 0.877 $\pm$ 0.025 & \textbf{1.000} $\pm$ 0.000\\ 
  \mtl      & 0.740 $\pm$ 0.095 & 0.894 $\pm$ 0.111 & \textbf{0.929} $\pm$ 0.079 & \textbf{0.960} $\pm$ 0.032 & \textbf{0.923} $\pm$ 0.127 & \textbf{1.000} $\pm$ 0.000 \\ \midrule

  \multicolumn{7}{c}{\textbf{NIFO Decision Support}}       \\ \midrule
  \resn     & 0.753 $\pm$ 0.056 & 0.899 $\pm$ 0.025 & 0.896 $\pm $0.044 & 0.897 $\pm$ 0.045 & 0.901 $\pm$ 0.025 & 0.929 $\pm$ 0.028 \\
  \tn       & 0.568 $\pm$ 0.049 & 0.775 $\pm$ 0.084 & 0.807 $\pm$ 0.038 & 0.868 $\pm$ 0.012 & 0.877 $\pm$ 0.025 & \textbf{1.000} $\pm$ 0.000\\ 
  \mtl      & \textbf{0.998} $\pm$ 0.032 & \textbf{0.995} $\pm$ 0.000 & \textbf{0.998} $\pm$ 0.000 & \textbf{0.998} $\pm$ 0.032 & \textbf{0.995} $\pm$ 0.000 & \textbf{0.999} $\pm$ 0.016 \\ \midrule

  \end{tabular}}
  \caption{Experiment results on VW double-square decision boundary . Models use 512-dimension embeddings; \mtl uses $\lambda=0.8$ and filtered triplets.}
  \label{tab:wv_square_filtered_l=0.8}
  \end{table}

%!TEX root = ../supp_main.tex

\begin{table}[t]
  \small
  \centering
    \resizebox{\textwidth}{!}{

  \begin{tabular}{@{}lrrrrrr@{}}
  \toprule
  Alignments   & 50\%   & 80\%   & 83\%  & 92\%  & 92.5\% & 100\%     \\ \midrule
  Weights  &  [0,0,1,1]  & [1,0,1,1] &  [0,1,1,1] &  [1,256,256,256] &  [256,1,256,256] &  [1,1,1,1]  \\ \midrule

  \multicolumn{7}{c}{\textbf{NI-H2H}} \\ \midrule
  \mtl vs. \resn     & 0.937 $\pm$ 0.021 & 0.917 $\pm$ 0.008 & 0.928 $\pm$ 0.029 & 0.915 $\pm$ 0.005 & 0.902 $\pm$ 0.050 & 0.896 $\pm$ 0.032\\ \midrule

  \multicolumn{7}{c}{\textbf{NO-H2H}} \\ \midrule
  \mtl vs. \resn     & 0.908 $\pm$ 0.130 & 0.985 $\pm$ 0.037 & 0.932 $\pm$ 0.008 & 0.943 $\pm$ 0.154 & 0.978 $\pm$ 0.032 & 0.953 $\pm$ 0.013\\ \midrule

  \multicolumn{7}{c}{\textbf{NINO Decision Support}}       \\ \midrule
  \resn     & 0.625 $\pm$ 0.258 & 0.783 $\pm$ 0.288 & 0.736 $\pm$ 0.286 & 0.738 $\pm$ 0.283 & 0.785 $\pm$ 0.289 & 0.803 $\pm$ 0.272 \\
  \tn       & 0.648 $\pm$ 0.413 & \textbf{0.914} $\pm$ 0.048 & 0.900 $\pm$ 0.127 & \textbf{0.940} $\pm$ 0.032 & 0.934 $\pm$ 0.111 & \textbf{1.000} $\pm$ 0.0000\\ 
  \mtl      & \textbf{0.720} $\pm$ 0.540 & 0.910 $\pm$ 0.159 & \textbf{0.935} $\pm$ 0.064 & 0.931 $\pm$ 0.079 & \textbf{0.939} $\pm$ 0.048 & 0.999 $\pm$ 0.016 \\ \midrule

  \multicolumn{7}{c}{\textbf{NIFO Decision Support}}       \\ \midrule
  \resn     & 0.682 $\pm$ 0.292 & 0.783 $\pm$ 0.309 & 0.783 $\pm$ 0.249 & 0.785 $\pm$ 0.254 & 0.784 $\pm$ 0.306 & 0.833 $\pm$ 0.197  \\
  \tn       & 0.531 $\pm$ 0.048 & 0.571 $\pm$ 2.875 & 0.575 $\pm$ 0.731 & 0.477 $\pm$ 0.064 & 0.425 $\pm$ 0.858 & 0.670 $\pm$ 0.381\\ 
  \mtl      & \textbf{0.996} $\pm$ 0.016 & \textbf{0.995} $\pm$ 0.000 & \textbf{0.998} $\pm$ 0.000 & \textbf{0.998 }$\pm$ 0.000 & \textbf{0.996} $\pm$ 0.016 & \textbf{0.996} $\pm$ 0.016 \\ \midrule
  \end{tabular}}
  \caption{Experiment results on VW double-square decision boundary . Models use 50-dimension embeddings; \mtl uses $\lambda=0.5$ and filtered triplets.}
  \label{tab:wv_square_filtered_l=0.5_d=50}
  \end{table}

% %!TEX root = ../supp_main.tex

\begin{table}[h!]
  \small
  \centering
  \begin{tabular}{@{}lrrrrrr@{}}
  \toprule
  Alignments   & 50\%   & 80\%   & 83\%  & 92\%  & 92.5\% & 100\%     \\ \midrule
  Weights  &  [0,0,1,1]  & [1,0,1,1] &  [0,1,1,1] &  [1,256,256,256] &  [256,1,256,256] &  [1,1,1,1]  \\ \midrule

  \multicolumn{7}{c}{\textbf{NI-H2H}} \\ \midrule
  \mtl vs. \resn     & 0.921 $\pm$ 0.015 & 0.900 $\pm$ 0.035 & 0.920 $\pm$ 0.023 & 0.895 $\pm$ 0.008 & 0.867 $\pm$ 0.034 & 0.846 $\pm$ 0.016\\ \midrule

  \multicolumn{7}{c}{\textbf{NO-H2H}} \\ \midrule
  \mtl vs. \resn     & 0.951 $\pm$ 0.034 & 0.969 $\pm$ 0.024 & 0.991 $\pm$ 0.002 & 0.991 $\pm$ 0.004 & 0.958 $\pm$ 0.010 & 0.980 $\pm$ 0.023\\ \midrule

  \multicolumn{7}{c}{\textbf{NINO Decision Support}}       \\ \midrule
  \resn     & \textbf{0.753} $\pm$ 0.056 & \textbf{0.899} $\pm$ 0.025 & \textbf{0.896} $\pm $0.044 & \textbf{0.897} $\pm$ 0.045 & \textbf{0.901} $\pm$ 0.025 & 0.929 $\pm$ 0.028 \\
  \tn       & 0.568 $\pm$ 0.049 & 0.775 $\pm$ 0.084 & 0.807 $\pm$ 0.038 & 0.868 $\pm$ 0.012 & 0.877 $\pm$ 0.025 & \textbf{1.000} $\pm$ 0.000\\ 
  \mtl      & 0.603 $\pm$ 0.051 & 0.801 $\pm$ 0.025 & 0.848 $\pm$ 0.053 & 0.880 $\pm$ 0.000 & 0.880 $\pm$ 0.081 & \textbf{1.000} $\pm$ 0.000 \\ \midrule

  \multicolumn{7}{c}{\textbf{NIFO Decision Support}}       \\ \midrule
  \resn     & 0.753 $\pm$ 0.056 & 0.899 $\pm$ 0.025 & 0.896 $\pm $0.044 & 0.897 $\pm$ 0.045 & 0.901 $\pm$ 0.025 & 0.929 $\pm$ 0.028 \\
  \tn       & 0.568 $\pm$ 0.049 & 0.775 $\pm$ 0.084 & 0.807 $\pm$ 0.038 & 0.868 $\pm$ 0.012 & 0.877 $\pm$ 0.025 & \textbf{1.000} $\pm$ 0.000\\ 
  \mtl      & \textbf{0.996} $\pm$ 0.004 & \textbf{0.999} $\pm$ 0.004 & \textbf{0.996} $\pm$ 0.004 & \textbf{0.996} $\pm$ 0.004 & \textbf{0.996} $\pm$ 0.004 & 0.997 $\pm$ 0.004 \\ \midrule
  \end{tabular}
  \caption{Experiment results on VW with \mtl useing $\lambda=0.5$ and unfiltered triplets. }
  \label{tab:wv_square_unfiltered_l=0.5}
  \end{table}

We compare \mtl, \resn, \tn on classification accuracy, triplet accuracy, and decision support performance for simulated humans. 
Table~\ref{tab:wv-square-clf-trip} shows how tuning $\lambda$ affects \mtl 's classification and triplet accuracy. Higher $\lambda$ drives \mtl to behave more simlar to \resn while lower \mtl is more similar to \tn. This shows that \mtl indeed learns both the classification task and human similarity prediction task.


In Table~\ref{tab:table1-ci} we present results with the best set of hyperparameter: filtered triplets, 512-dimension embedding, $\lambda=0.5$. We present our main observations as follows.


\para{\mtl significantly outperforms \resn in H2H.} 
Our synthetic humans prefer \mtl over \resn by a large margin as justifications for both nearest in-class examples and nearest out-of-class examples, indicating the NIs and NOs selected from the \mtl representations are more aligned with the synthetic humans than \tn. 

For NI H2H, the preference towards \mtl declines as the alignment improves, because if alignment between human similarity and classification increases, \resn can capture human similarity as a byproduct of classification.
Also, NO H2H is higher than NI H2H, suggesting \resn learns a better representation within each class compared to between classes. 

\para{\mtl provides the best decision support.} 
Table \ref{tab:table1-ci} shows that \mtl achieves the highest NINO and NIFO decision support scores in all alignments. 
In NINO decision support, \resn consistently outperforms \tn, highlighting that representation solely learned for metric learning is ineffective for decision support.
For all models, the decision support performance improves as the alignments increases, suggesting that decision support is easier when human similarity judgement is aligned with the classification task.
\resn and \tn are more comparable in NIFO, while \mtl consistently shows 100\%.
The fact that \resn shows comparable performance between NINO and NIFO further confirms that \resn does not capture meaning similarity for examples from two different classes.

\para{Filtering triplets leads to better decision support}. \figref{fig:square-filter}(a) shows that filtering class-inconsistent triplets improves \mtl's decision support performance across all alignments. However, filtering does lead to slightly worse H2H performance. This suggests that in terms of decision support, the benefit of filtering out human noise may overweigh the loss of some similarity judgment.



\para{The effect of noise in triplets.}
To test the limitations of \mtl, we perturb human judgment by adding noise to triplets. 
We add noise to triplets by randomly flipping the similarity judgement (i.e., a triplet $(x^r, x^+, x^-)$ becomes $(x^r, x^-, x^+)$) with probability $p$.
As shown in \figref{fig:noise}, direct comparison results decreases linearly as noise increases, decision support performance does not start decreasing util $p=0.5$. 
This is not as surprising for two reasons. First, since VW is a binary dataset, half of the triplets are ones where $x^+, x^-$ belong to the same class; 
flipping these triplets does not have a significant effect on the generated embeddings. 
More importantly, filtering greatly reduces the adversarial effect of adding noise, highlighting the importance of filtering. 
In comparison, \tn drops linearly in NINO. \mtl trained on unfiltered triplets is also more vulnerable to noise perturbations. H2H and decision support performance is overall worse and decreases much faster than that of filtered \mtl.


\para{Number of triplets} In ~\figref{fig:vary-num} we examine the effect of the number of triplets. We decrease number of triplets by powers of 2 
and find that H2H preference towards \mtl indeed declines as \mtl representation is less human-compatible with fewer training data.
As for decision support, in NINO, \mtl declines and eventually approaches \resn except the outlier at end, while in NIFO, \mtl is able to stay 100\% even as the number of triplets declines. 


\subsubsection{Hyperparameter Tuning}

\paragraph{Results for different $\lambda$.}
We show results for $\lambda=0.2$ in Table~\ref{tab:wv_square_filtered_l=0.2} and $\lambda=0.8$ in Table~\ref{tab:wv_square_filtered_l=0.8}. We do not observe a clear trend between $\lambda$ and evaluation metric performances. In the previous section we present \mtl with $\lambda=0.5$ as it shows best overall performance.

\paragraph{Results for 50-dimension embedding.} Table~\ref{tab:wv_square_filtered_l=0.5_d=50} shows results with all models outputing a 50-dimension embedding. Similar to $\lambda$, 50-dimension embedding gives slightly worse resutls, but the trend between embedding dimension and our evaluation metrics is not clear.

\paragraph{Results for unfiltered \mtl.} 
In table~\ref{tab:wv_square_unfiltered_l=0.5} we show results on \mtl with $\lambda=0.5$ using unfiltered triplets. As cocncluded previously, an unfiltered \mtl performs much worse in NINO decision support but slightly better in H2H.

 
  \begin{figure}[t!]
    \center
    \includegraphics[width=0.31\textwidth]{figures/supp/linear_dist.png}
    \caption{VW dataset with a linear decision boundary.}
    \label{fig:linear-dist}
  \end{figure}


\begin{table}[t]
  \small
  \centering
  \begin{tabular}{@{}lrrrr@{}}
  \toprule
  Alignments   & 56\%   & 84\%   & 95\%  & 98.5\%    \\ \midrule
  Weights  &  [0,1,1,1]  & [1,0,1,1] &  [1,1,1,1] &  [32,256,1,1]  \\ \midrule

  \multicolumn{5}{c}{\textbf{NI-H2H}} \\ \midrule
  \mtl vs. \resn     & 0.913 $\pm$ 0.023 & 0.922 $\pm$ 0.008 & 0.899 $\pm$ 0.020 & 0.848 $\pm$ 0.055\\ \midrule

  \multicolumn{5}{c}{\textbf{NO-H2H}} \\ \midrule
  \mtl vs. \resn     & 0.932 $\pm$ 0.034 & 0.960 $\pm$ 0.027 & 0.921 $\pm$ 0.013 & 0.928 $\pm$ 0.034\\ \midrule

  \multicolumn{5}{c}{\textbf{NINO Decision Support}}       \\ \midrule
  \resn     & 0.778 $\pm$ 0.084 & 0.792 $\pm$ 0.144 & 0.839 $\pm$ 0.130 & 0.927 $\pm$ 0.019 \\
  \tn       & 0.554 $\pm$ 0.175 & 0.770 $\pm$ 0.318 & 0.950 $\pm$ 0.095 & 0.914 $\pm$ 0.075\\ 
  \mtl      & \textbf{0.841} $\pm$ 0.053 & \textbf{0.911} $\pm$ 0.053 & \textbf{0.967} $\pm$ 0.009 & \textbf{0.961} $\pm$ 0.014 \\ \midrule

  \multicolumn{5}{c}{\textbf{NIFO Decision Support}}       \\ \midrule
  \resn     & 0.802 $\pm$ 0.249 & 0.815 $\pm$ 0.151 & 0.848 $\pm$ 0.188 & 0.953 $\pm$ 0.051 \\
  \tn       & 0.473 $\pm$ 1.016 & 0.653 $\pm$ 1.747 & 0.441 $\pm$ 0.016 & 0.381 $\pm$ 0.474\\ 
  \mtl      & \textbf{0.979} $\pm$ 0.014 & \textbf{0.977} $\pm$ 0.009 & \textbf{0.977} $\pm$ 0.009 & \textbf{0.978} $\pm$ 0.013 \\ \midrule
  \end{tabular}

  \caption{Experiment results on VW linear decision boundary data. Models use 512-dimension embeddings; \mtl uses $\lambda=0.5$ and filtered triplets.}
  \label{tab:wv_lin_filtered_l=0.5}
  \end{table}


\subsection{Linear decision boundary experiment results}
We create a linear decision boundary on VW without altering the dataset (\figref{fig:linear-dist}). We find the results are overall similar to the original VW data.

\subsubsection{Results}

\paragraph{H2H and decision support results}
In Table~\ref{tab:wv_lin_filtered_l=0.5} we present results with the best set of hyperparameter: filtered triplets, 512-dimension embedding, $\lambda=0.5$. Similar to the experiment on VW double-square decision boundary experiments, we observe that \mtl outperforms \resn in both H2H comparison and decision support performance. 

We show results for $\lambda=0.2$ in Table~\ref{tab:wv_lin_filtered_l=0.2} and $\lambda=0.8$ in Table~\ref{tab:wv_lin_filtered_l=0.8}. We also show results with 50-dimension \mtl in Table~\ref{tab:wv_lin_filtered_l=0.5_d=50} and unfiltered \mtl in Table~\ref{tab:wv_lin_unfiltered_l=0.5}. 
Similar to the experiment on VW double-square decision boundary experiments, we see no clear relation between $\lambda$, embedding dimension and our evaluation metrics. Table 16 again shows that filtered \mtl leads to better decision support but worse H2H.







\begin{table}[t]
  \small
  \centering
  \begin{tabular}{@{}lrrrr@{}}
  \toprule
  Alignments   & 56\%   & 84\%   & 95\%  & 98.5\%    \\ \midrule
  Weights  &  [0,1,1,1]  & [1,0,1,1] &  [1,1,1,1] &  [32,256,1,1]  \\ \midrule

  \multicolumn{5}{c}{\textbf{NI-H2H}} \\ \midrule
  \mtl vs. \resn     & 0.936 $\pm$ 0.024 & 0.921 $\pm$ 0.008 & 0.912 $\pm$ 0.074 & 0.856 $\pm$ 0.034\\ \midrule

  \multicolumn{5}{c}{\textbf{NO-H2H}} \\ \midrule
  \mtl vs. \resn     & 0.946 $\pm$ 0.032 & 0.974 $\pm$ 0.032 & 0.949 $\pm$ 0.003 & 0.934 $\pm$ 0.029\\ \midrule

  \multicolumn{5}{c}{\textbf{NINO Decision Support}}       \\ \midrule
  \resn     & 0.778 $\pm$ 0.084 & 0.792 $\pm$ 0.144 & 0.839 $\pm$ 0.130 & 0.927 $\pm$ 0.019 \\
  \tn       & 0.554 $\pm$ 0.175 & 0.770 $\pm$ 0.318 & 0.950 $\pm$ 0.095 & 0.914 $\pm$ 0.075\\ 
  \mtl      & \textbf{0.845} $\pm$ 0.127 & \textbf{0.880} $\pm$ 0.127 & \textbf{0.956} $\pm$ 0.016 & \textbf{0.956} $\pm$ 0.111 \\ \midrule

  \multicolumn{5}{c}{\textbf{NIFO Decision Support}}       \\ \midrule
  \resn     & 0.802 $\pm$ 0.249 & 0.815 $\pm$ 0.151 & 0.848 $\pm$ 0.188 & 0.953 $\pm$ 0.051 \\
  \tn       & 0.473 $\pm$ 1.016 & 0.653 $\pm$ 1.747 & 0.441 $\pm$ 0.016 & 0.381 $\pm$ 0.474\\ 
  \mtl      & \textbf{0.974} $\pm$ 0.016 & \textbf{0.970} $\pm$ 0.064 & \textbf{0.968} $\pm$ 0.064 & \textbf{0.988} $\pm$ 0.032 \\ \midrule
  \end{tabular}
  \caption{Experiment results on VW linear decision boundary data. Models use 512-dimension embeddings; \mtl uses $\lambda=0.2$ and filtered triplets.}

  \label{tab:wv_lin_filtered_l=0.2}
  \end{table}

% 




\begin{table}[h]
  \small
  \centering
  \begin{tabular}{@{}lrrrr@{}}
  \toprule
  Alignments   & 56\%   & 84\%   & 95\%  & 98.5\%    \\ \midrule
  Weights  &  [0,1,1,1]  & [1,0,1,1] &  [1,1,1,1] &  [32,256,1,1]  \\ \midrule

  \multicolumn{5}{c}{\textbf{NI-H2H}} \\ \midrule
  \mtl vs. \resn     & 0.906 $\pm$ 0.122 & 0.909 $\pm$ 0.111 & 0.882 $\pm$ 0.135 & 0.848 $\pm$ 0.050\\ \midrule

  \multicolumn{5}{c}{\textbf{NO-H2H}} \\ \midrule
  \mtl vs. \resn     & 0.926 $\pm$ 0.021 & 0.955 $\pm$ 0.199 & 0.936 $\pm$ 0.053 & 0.912 $\pm$ 0.095\\ \midrule

  \multicolumn{5}{c}{\textbf{NINO Decision Support}}       \\ \midrule
  \resn     & 0.778 $\pm$ 0.084 & 0.792 $\pm$ 0.144 & 0.839 $\pm$ 0.130 & 0.927 $\pm$ 0.019 \\
  \tn       & 0.712 $\pm$ 0.077 & 0.833 $\pm$ 0.051 & 0.942 $\pm$ 0.009 & 0.963 $\pm$ 0.013\\ 
  \mtl      & 0.824 $\pm$ 0.175 & 0.895 $\pm$ 0.159 & 0.950 $\pm$ 0.032 & 0.969 $\pm$ 0.016 \\ \midrule

  \multicolumn{5}{c}{\textbf{NIFO Decision Support}}       \\ \midrule
  \resn     & 0.802 $\pm$ 0.249 & 0.815 $\pm$ 0.151 & 0.848 $\pm$ 0.188 & 0.953 $\pm$ 0.051 \\
  \tn       & 0.539 $\pm$ 0.155 & 0.460 $\pm$ 0.397 & 0.571 $\pm$ 0.205 & 0.586 $\pm$ 0.577\\ 
  \mtl      & 0.981 $\pm$ 0.048 & 0.964 $\pm$ 0.206 & 0.961 $\pm$ 0.175 & 0.978 $\pm$ 0.064 \\ \midrule
  \end{tabular}
  \caption{Experiment results on VW linear decision boundary data. \mtl uses $\lambda=0.8$.}
  \label{tab:wv_lin_filtered_l=0.8}
  \end{table}

\input{tables/supp/wv_squarelin_filtered_l=0.5_d=50.tex}

% %!TEX root = ../supp_main.tex

\begin{table}[t]
  \small
  \centering
  \begin{tabular}{@{}lrrrr@{}}
  \toprule
  Alignments   & 56\%   & 84\%   & 95\%  & 98.5\%    \\ \midrule
  Weights  &  [0,1,1,1]  & [1,0,1,1] &  [1,1,1,1] &  [32,256,1,1]  \\ \midrule

  \multicolumn{5}{c}{\textbf{NI-H2H}} \\ \midrule
  \mtl vs. \resn     & 0.942 $\pm$ 0.008 & 0.933 $\pm$ 0.122 & 0.903 $\pm$ 0.016 & 0.880 $\pm$ 0.026\\ \midrule

  \multicolumn{5}{c}{\textbf{NO-H2H}} \\ \midrule
  \mtl vs. \resn     & 0.977 $\pm$ 0.034 & 0.988 $\pm$ 0.050 & 0.962 $\pm$ 0.048 & 0.970 $\pm$ 0.085\\ \midrule

  \multicolumn{5}{c}{\textbf{NINO Decision Support}}       \\ \midrule
  \resn     & \textbf{0.778} $\pm$ 0.084 & 0.792 $\pm$ 0.144 & 0.839 $\pm$ 0.130 & 0.927 $\pm$ 0.019 \\
  \tn       & 0.554 $\pm$ 0.175 & 0.770 $\pm$ 0.318 & 0.950 $\pm$ 0.095 & 0.914 $\pm$ 0.075\\ 
  \mtl      & 0.700 $\pm$ 0.572 & \textbf{0.820} $\pm$ 0.159 & \textbf{0.955} $\pm$ 0.095 & \textbf{0.962} $\pm$ 0.000 \\ \midrule

  \multicolumn{5}{c}{\textbf{NIFO Decision Support}}       \\ \midrule
  \resn     & 0.802 $\pm$ 0.249 & 0.815 $\pm$ 0.151 & 0.848 $\pm$ 0.188 & 0.953 $\pm$ 0.051 \\
  \tn       & 0.473 $\pm$ 1.016 & 0.653 $\pm$ 1.747 & 0.441 $\pm$ 0.016 & 0.381 $\pm$ 0.474\\ 
  \mtl      & \textbf{0.978} $\pm$ 0.064 & \textbf{0.969} $\pm$ 0.048 & \textbf{0.961} $\pm$ 0.016 & \textbf{0.975} $\pm$ 0.000 \\ \midrule
  \end{tabular}
  \caption{Experiment results on VW linear decision boundary data. Models use 512-dimension embeddings; \mtl uses $\lambda=0.5$ and unfiltered triplets.}

  \label{tab:wv_lin_unfiltered_l=0.5}
  \end{table}

\paragraph{The effect of perturbations} Similar to previous results, adding noise harms H2H and decision support performance with unfiltered \mtl more so than filtered (~\figref{fig:lin-noise-filtered}, ~\figref{fig:lin-noise-unfiltered}). ~\figref{fig:lin-num} shows H2H and decision support performance with decreasing number of triplets; results are similar to VW double-square decision boundary data.

In conclusion, from two sets of synthetic data with different decision boundaries, we see \mtl overall outperforms our baselines \resn and \tn in H2H and decision support.

\begin{figure}[t]
    \centering
    \begin{subfigure}[b]{0.32\textwidth}
      \includegraphics[width=\textwidth]{figures/supp/wv_squarelin_noise_filtered_h2h.pdf}
      \end{subfigure}
      \begin{subfigure}[b]{0.32\textwidth}
        \includegraphics[width=\textwidth]{figures/supp/wv_squarelin_noise_filtered_NINO.pdf}
      \end{subfigure}
      \begin{subfigure}[b]{0.32\textwidth}
        \includegraphics[width=\textwidth]{figures/supp/wv_squarelin_noise_filtered_NIFO.pdf}
      \end{subfigure}
      \caption{Results on VW linear decision boundary data with varying noise levels. \mtl uses filtered triplets.}
      \label{fig:lin-noise-filtered}
\end{figure}


\begin{figure}[h!]
    \centering
    \begin{subfigure}[b]{0.32\textwidth}
      \includegraphics[width=\textwidth]{figures/supp/wv_squarelin_noise_unfiltered_h2h.pdf}
      \end{subfigure}
      \begin{subfigure}[b]{0.32\textwidth}
        \includegraphics[width=\textwidth]{figures/supp/wv_squarelin_noise_unfiltered_NINO.pdf}
      \end{subfigure}
      \begin{subfigure}[b]{0.32\textwidth}
        \includegraphics[width=\textwidth]{figures/supp/wv_squarelin_noise_unfiltered_NIFO.pdf}
      \end{subfigure}
      \caption{Results on VW linear decision boundary data with varying noise levels. \mtl uses unfiltered triplets.}
      \label{fig:lin-noise-unfiltered}
\end{figure}


\begin{figure}[h!]
    \centering
    \begin{subfigure}[b]{0.32\textwidth}
      \includegraphics[width=\textwidth]{figures/supp/wv_squarelin_num_h2h.pdf}
      \end{subfigure}
      \begin{subfigure}[b]{0.32\textwidth}
        \includegraphics[width=\textwidth]{figures/supp/wv_squarelin_num_NINO.pdf}
      \end{subfigure}
      \begin{subfigure}[b]{0.32\textwidth}
        \includegraphics[width=\textwidth]{figures/supp/wv_squarelin_num_NIFO.pdf}
      \end{subfigure}
      \caption{Results on VW linear decision boundary data with varying number of triplets. \mtl uses filtered triplets.}
      \label{fig:lin-num}
\end{figure}





%!TEX root = main.tex 

\section{Broader Impacts and Potential Negative Societal Impacts}
Although coming from a genuine goal to improve human-AI collaboration by aligning AI models with human intuition, our work may have potential negative impacts for the society. We discuss these negative impacts from two perspectives: the multi-task learning framework and the decision support policies.


\subsection{Multi-task learning framework}
Our \mtl models are trained with two sources of data. The first source of data is classification annotations where groundtruth maybe be derived from scientific evidence or crowdsourcing with objective rules or guidelines. The second source of data is human judgment annotations where groundtruth is probably always acquired from crowdworkers with subjective perceptions. When our data is determined with subjective perceptions, the model that learns from it may inevitably develop bias based on the sampled population. If not carefully designed, the human judgment dataset may contain bias against certain minority group depending on the domain and the task of the dataset. For example, similarity judgment based on chest xray of patients in one gender group or racial group may affect the generalizability of the representations learned from it, and may lead to fairness problems in downstream tasks. It is important for researchers to audit the data collection process and make efforts to avoid such potential problems.


\subsection{Decision support policies}
Among a wide variety of example selection policies, our policies to choose the decision support examples are only attempts at leveraging AI model representations to increase human performance. 
We believe that they are reasonable strategies for evaluating representations learned by a model, but future work is required to establish their use in practice.
% They are by no means estabalished and standardized ways to evaluate the effectness of representation and the interaction design. 

The NINO policy aims to select the nearast examples in each class, therefore limiting the decision problem to a small region around the test example. We hope this policy allow human users to zoom in the local neighborhood and scrutinize the difference between the relatively close examples. In other words, NINO help human users develop a local decision boundary with the smallest possible margin. This could be useful for confusing test cases that usually require careful examinations. However, the NINO policy adopts an intervention to present a small region in the dataset and may downplay the importance of global distribution in human users' decision making process. 

The NIFO policy aim to select the nearest in-class examples but the furthest out-of-class examples. It aims to maximize the visual difference between examples in opposite class, thus require less effort for human users to adopt case-based reasoning for classification. It also helps human users to develop a local decision boundary with the largest possible margin. However, when model prediction is incorrect, the policy end up selecting the furthest in-class examples with the nearest out-of-class examples, completely contrary to what it is design to do, may lead to even over-reliance or even adversarial supports.

In general, decision support policies aims to choose a number of supporting examples without considering some global properties such as representativeness and diversity. While aiming to reduce humans' effort required in task by encouraging them to make decision in a local region, the decision support examples do not serve as a representative view of the whole dataset, and may bias human users to have a distorted impression of the data distribution. It remains an open question that how to ameliorate these negative influence when designing decision support interactions with case-based reasoning.


% You can have as much text here as you want. The main body must be at most $4$ pages long.
% For the final version, one more page can be added.
% If you want, you can use an appendix like this one, even using the one-column format.
%%%%%%%%%%%%%%%%%%%%%%%%%%%%%%%%%%%%%%%%%%%%%%%%%%%%%%%%%%%%%%%%%%%%%%%%%%%%%%%
%%%%%%%%%%%%%%%%%%%%%%%%%%%%%%%%%%%%%%%%%%%%%%%%%%%%%%%%%%%%%%%%%%%%%%%%%%%%%%%


\end{document}


% This document was modified from the file originally made available by
% Pat Langley and Andrea Danyluk for ICML-2K. This version was created
% by Iain Murray in 2018, and modified by Alexandre Bouchard in
% 2019 and 2021 and by Csaba Szepesvari, Gang Niu and Sivan Sabato in 2022. 
% Previous contributors include Dan Roy, Lise Getoor and Tobias
% Scheffer, which was slightly modified from the 2010 version by
% Thorsten Joachims & Johannes Fuernkranz, slightly modified from the
% 2009 version by Kiri Wagstaff and Sam Roweis's 2008 version, which is
% slightly modified from Prasad Tadepalli's 2007 version which is a
% lightly changed version of the previous year's version by Andrew
% Moore, which was in turn edited from those of Kristian Kersting and
% Codrina Lauth. Alex Smola contributed to the algorithmic style files.
