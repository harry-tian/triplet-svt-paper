%!TEX root = main.tex 

\section{Experimental Setup}

\paragraph{Triplet annotations}

We collect triplet annotations on a natural image dataset: Butterflies v.s. Moths (BM). Following \citet{singla2014near}, we acquired 120 images from the publicly available dataset set (ImageNet \cite{krizhevsky2012imagenet}). BM is a binary classification problem and each class contains two species.

We recruit 80 crowdworkers on Prolific to acquire 2400 visual similarity triplets. In each question for the annotator, we show the reference image on top and two candidate below, and ask 2-Alternative-Forced-Choice (2AFC) question: which candidate image looks more similar to the reference image. 

\paragraph{Distance matrix construction}
We experiment with the following three distance matrix construction methods. The distance matrix $D\in\mathbb{R}^{n\times n}$ is always initialized with all zeros.

\begin{itemize}[itemsep=0pt, topsep=0pt, leftmargin=*]
    \item Increment/decrement by 1 (\textbf{ID1}). For all triplets $t:=(i,j,k)\in T$, increment $D_{i,j}$, $D_{j,i}$ and decrement $D_{i,k}$, $D_{k,i}$ by 1. Then, fill the diagonals of $D$ with $2\cdot min(D)$.
    
    \item Increment/decrement by random (\textbf{IDR}). For all triplets $t:=(i,j,k)\in T$, increment $D_{i,j}$, $D_{j,i}$ and decrement $D_{i,k}$, $D_{k,i}$ by a random value in (0,1). Then, fill the diagonals of $D$ with $2\cdot min(D)$.
    
    \item always increment  (\textbf{INC}). For all triplets $t:=(i,j,k)\in T$, increment $D_{i,j}$, $D_{j,i}$ by 1 and increment $D_{i,k}$, $D_{k,i}$ by 2. Then, fill the diagonals of $D$ with 0.
    
\end{itemize}

\paragraph{Singular Value Thresholding} We use the constructed distance matrix $D$ along with a mask $M$ as the input to SVT. $M\in\mathbb{R}^{n\times n}$ has 1s in indices filled by the distance matrix construction and 0s elsewhere. For each $D$ and  $M$, we find the best hyperparameters for SVT through bayesian optimization on $\tau$, the singular value thresholding amount, and $\epsilon$, the convergence condition.

\paragraph{Evaluations} 
We randomly split the triplets into a training set and a test set with a 0.8:0.2 ratio. We evaluate a distance matrix by triplet accuracy: the percentage of triplets in the test set that the distance matrix agrees with.

We also experiment our methods robustness to varying number of triplets. We decrease the number of triplets by 10\% to create 10 settings and measure the test triplet accuracy in each setting.
